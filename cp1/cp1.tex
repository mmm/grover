\documentclass[titlepage,12pt]{utarticle}

\usepackage{amsfonts}
\usepackage{amssymb}
\usepackage{epsfig}

\include{mmm_macros}

\begin{document}


%---------------------------------------------------------------------------------------
%---------------------------------------------------------------------------------------
%---------------------------------------------------------------------------------------
%---------------------------------------------------------------------------------------
\section{Geodesics of $\mathbb{C}P^1$}\label{sec:geodesics_cp1}

Grover's algorithm is claimed\cite{9910115} to follow a geodesic of
the Fubini--Study metric on $\mathbb{C}P^n$.
I going to try to model the geodesics of Fubini--Study numerically.

For $\mathbb{C}P^1$, the Fubini--Study metric takes
the form\cite{Nash}
\begin{equation}
ds^2 = \frac{i}{2} \frac{1}
                        {(1 + |z|^2)^2}
       dz\otimes d{\bar z}.
\label{FSmetric}
\end{equation}

Geodesics obey 
\begin{equation}
{\ddot z}^i + \Gamma^i_{jk}{\dot z}^j{\dot{\bar z}}^k = 0.
\end{equation}
(also Hermitian conjugate).

For $\mathbb{C}P^1$, the only Christoffel symbol
(again, also the Hermitian conjugate) is
\begin{equation}
\Gamma^1_{11} = - \frac{2{\bar z}}
                       {(1 + |z|^2)},
\end{equation}
so the geodesic equations of motion for $\mathbb{C}P^1$
become
\begin{equation}
{\ddot z} - \frac{2{\bar z}}
                   {(1 + |z|^2)}
              {\dot z}^2 = 0.
\label{geodesics}
\end{equation}


\subsection{Numerical stuff}

The immediate goal here is to solve for these geodesics
numerically.  I'd like to talk about stability of solns
and fixed point analysis when noise is introduced, so
can anything like that be said here?  This may be too easy
of an equation.  We'll see\dots

The equation of motion for the geodesic (\ref{geodesics})
can easily be reduced to the following
system of first order equations
\begin{align}
{\dot z} &= v\\
{\dot v} &= \frac{2{\bar z}}
                   {(1 + |z|^2)}
              v^2.
\label{firstorder}
\end{align}
(also with Hermitian conjugates).

Numerically, lets try to integrate these four equations
using some sort of simple Runge--Kutta method.

\subsubsection{Runge--Kutta}


I thought breifly about trying to find an object framework
in which to cast the numerical integration of a differential
equation.  It seems like you have a specialized CP1Geodesic
that IS-A ODE.  The Runga--Kutta step is generic, you just
need to specialize the derivative for each equation.  The problem
with this is you want to run the rk step on an array of
equations, not have each equation 
\begin{verbatim}
->rkstep() 
\end{verbatim}
or something.
Hmmm... I think I'm just gonna quick--n--dirty ``C'' the thing
as a ``prototype'' (grin).



\subsubsection{Fixed point analysis}

It seems that the system of equations (\ref{firstorder})
has a single (?) fixed point at the origin.
Need to linearize (\ref{firstorder}) around this
fixed point\dots by Tay
\begin{align}
{\dot z} &= v\\
{\dot v} &= \frac{2{\bar z}}
                   {(1 + |z|^2)}
              v^2.
\label{firstorder}
\end{align}





























%\lemma This sucks. This really sucks. to the $\tr\rho$, or
%maybe $\tr{\rho}$, or maybe even $\tr[\rho]$.  Is this $\C$, or
%does $\ket{i}$.
%
%\proof This is the proof.\qed

%---------------------------------------------------------------------------------------
%---------------------------------------------------------------------------------------
\section{Geometrical Aspects of Quantum Information Theory} 

%-------------------------------------------------------------

\subsection{Spaces of States}

%----------------------------------------------------------------------------------
%-------------------------------------------------------------------------------------

\section{Quantum Coding}
\label{sec:quantumCoding}

%-------------------------------------------------------------

\subsection{The Code Bundle}

%-------------------------------------------------------------

\subsection{Quantum Error Correction}

%--------------------------------------------------------------------------------------
%-----------------------------------------------------------------------------------
\section{Summary and Conclusions} 


%---------------------------------------------------------------------------------------
%---------------------------------------------------------------------------------------

\begin{thebibliography}{99}

%\bibitem{Preskill}John Preskill (1999) ``Quantum information and physics: some
%future directions,'' {\tt quant-ph/9904022}.

\bibitem{me}Mark Byrd, 
 ``Differential Geometry of $SU(3)$ with Applications to 3 state Systems'', 
{\it J. Math. Phys.} {\bf 39} (11) (1998) 6125-6136.

\bibitem{Slatert}We want to thank Paul Slater for prompting this important 
realization.

\bibitem{Dittmann}J. Dittmann, ``Note on Explicit Formula for the Bures Metric,'' quant-ph/9808044.

\bibitem{9910115}J.J. Alvarez and C. G\'omez, ``A comment on Fisher Information and Quantum
Algorithms,'' quant-ph/9910115.

\bibitem{Nash}C. Nash, (1991) {\it Differential Topology and Quantum Field Theory}, 
Academic Press, San Diego.

\end{thebibliography}

%---------------------------------------------------------------------------------------
%---------------------------------------------------------------------------------------

\end{document}
